% !TEX root = ./main.tex
% !TEX encoding = UTF-8 Unicode
% !TEX program = pdflatex
% !TeX spellcheck = it_IT

\graphicspath{{Immagini/},{Immagini/webServer/}}

\chapter{Web Server}

\section{Traccia}
Emulare un Web Server Apache ed un set di utenti che richiedono delle risorse.
Monitorare il workload osservato, successivamente caratterizzarlo ed effettuare
un'analisi di performance.\\

\section{Piattaforma}
Il client su cui è stato eseguito \textbf{JMeter}(un generatore di workload), è
un notebook MSI con le seguenti caratteristiche:

\begin{itemize}
  \item \textbf{Processore}: Intel(R) Core(TM) i7-7700HQ @ 2.80GHz
  \item \textbf{Memoria Ram}: 16GB DDR4-2400MHz
  \item \textbf{Tipo sistema}: Windows 10 64bit, processore basato su x64
  \item \textbf{Storage}: SSD Kingston M.2.SATA 480GB
\end{itemize}

Il Web Server apache è stato installato su un notebook Asus con le seguenti
caratteristiche:

\begin{itemize}
  \item \textbf{Processore}: Intel(R) Core(TM) Pentium
  \item \textbf{Memoria Ram}: 4GB DDR3-1600MHz
  \item \textbf{Tipo sistema}: Ubuntu 16.4 LTS, processore basato su x64
  \item \textbf{Storage}: SSD Samsung 850 EVO SATA 3
\end{itemize}

Infine la connessione Client-Server è stata effettuata in modo diretto tramite un
cavo ethernet.\\

\section{Performance Analysis}
Per caratterizzare il Workload sono state simulate richieste HTTP random al server,
teli richieste sono state effettuate su 16 pagine html suddivise in dimensioni:

\begin{itemize}
  \item \textbf{\textit{Picccole}}: dell'ordine delle decine di KB;
  \item \textbf{\textit{Medie}}: dell'ordine delle centinaia di KB;
  \item \textbf{\textit{Grandi}}: dell'ordine delle migliaia di KB;
\end{itemize}

Per la simulazione dei client è stato utilizzato il tool \textbf{JMeter}

\begin{figure}[!htbp]
  \includegraphics[width=1\linewidth,keepaspectratio]{logo_jmeter}
\end{figure}

\section{Analisi alto livello}

Il carico da sottoporre al server è stato generato tramite il tool JMeter.\\
