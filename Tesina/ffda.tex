% !TEX root = ./main.tex
% !TEX encoding = UTF-8 Unicode
% !TEX program = pdflatex
% !TeX spellcheck = it_IT

\graphicspath{{Immagini/},{Immagini/ffda/}}

\chapter{Field Failure Data Analysis}
La \textbf{FFDA} è effettuata sui dati relativi ad un sistema in
esercizio(\textit{on Field}), al fine di scoprire i possibili fallimenti.\\
L'approccio utilizzato nella FFDA prevede di monitorare il sistema in esecuzione,
misurando parametri di dependability e prelevando tutte le possibili informazioni
sul fallimenti rilevati.\\
Un fallimento è una deviazione del comportamento del sistema dal suo corretto e
specifico funzionamento.\\
La metodologia FFDA è basata su tre fasi fondamentali:
\begin{itemize}
  \item \textbf{Data Logging \& Collection} - consiste nella definizione di cosa
  collezionare e come farlo;
  \item \textbf{Data Filtering \& Manipulation} - consiste nella manipolazione ù
  dei dati collezionati, utilizzando \textit{filtering} e \textit{coalescence};
  \item \textbf{Data Logging \& Collection} - consiste nell'esecuzione di un'analisi
  statistica su dati precedentemente manipolati, per calcolarne misure quantitative
  e identificarne possibili trends.\\
\end{itemize}

\section{Traccia}
Dati i due file di log \textit{MercuryErrorLog} e \textit{BGLErrorLog},
preventivamente filtrati per ottenere solo eventi di fallimento, si vuole:
\begin{itemize}
  \item determinare la finestra di coalescenza;
  \item raggruppare tutte le entry appartenenti alle stesse finestre di
  coalescenza(tuple);
  \item ricavare la \textit{CDF} del \textbf{TTF}(Time-To-Failure) e della
  \textbf{Reliability Empirica};
  \item fitting della CDF della reliability empirica provando tutti i modelli
  studiati (\textit{Esponenziale}, \textit{Weibull} ed \textit{Iperesponenziale});
  \item utilizzare il test di \textit{Kolmogorov-Smirnov} per controllare che il
  modello ipotizzato fitti adeguatamente i dati.\\
\end{itemize}

\clearpage

\section{Mercury}
Il sistema Mercury consiste di nodi IBM.\\
Il cluster ha un'architettura a 3 livelli(nodi \textit{login}, \textit{computation}
e \textit{storage}) ed un solo nodo di management
(\textit{tg-master}).\\
Il log è formato dai seguenti campi:
\begin{itemize}
  \item \textbf{Timestamp};
  \item \textbf{Nodo Origine};
  \item \textbf{Categoria Errore}:
  \begin{itemize}
    \item DEV, PRO, MEM, NET, IO, OTH;
  \end{itemize}
  \item \textbf{Messaggio}.
\end{itemize}

\subsection{Finestra di Coalescenza - CWIN}
La finestra di coalescence, definita in secondi, definisce un intervallo temporale
in cui cadono tutti gli eventi che vi appartengono.\\
Lo script \textbf{tupleCount\_func\_CWINpy.sh}(riportato in figura \ref{coalescence_window_mercury}), con input i file \textit{MercuryErrorLog.txt}
e \textit{tentative-CWIN.txt}, è stato utilizzato per calcolare il numero di tuple al
variare della dimensione della finestra di coalescenza.\\

\begin{figure}[!htbp]
  \includegraphics[width=1\linewidth,keepaspectratio]{coalescence_window_mercury}
  \caption{Script conteggio tuple al variare della finestra di coalescenza}
  \label{coalescence_window_mercury}
\end{figure}

\clearpage

L'output di tale script, plottato in matlab e presente in figura \ref{plot_coalescence_window_mercury},
è infine utilizzato per determinare un singolo valore di CWIN, ottenuto
considerando il punto successivo al ginocchio(knee) della curva.\\

\begin{figure}[!htbp]
  \includegraphics[width=1\linewidth,keepaspectratio]{plot_coalescence_window_mercury}
  \caption{Plot finestre di coalescenza}
  \label{plot_coalescence_window_mercury}
\end{figure}

%%ANDREA è ARRIVATO QUI!!!
