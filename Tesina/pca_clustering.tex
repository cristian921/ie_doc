% !TEX root = ./main.tex
% !TEX encoding = UTF-8 Unicode
% !TEX program = pdflatex
% !TeX spellcheck = it_IT

\chapter{PCA e Clustering}
Estrapolare un Workload sintetico a partire dal workload reale riportato nel file
 \textit{PCA-CLASTERING-2017.jmp}.

\section{Obiettivo}
Considerato il workload reale si vuole ottenere un workload sintetico
che contenga un numero di osservazioni minori ma che conservando quanta
più varianza possibile.

\section{Estrazione del Workload Sintetico}
Per l'estrazione del workload sintetico, dopo aver visionato i dati, si è scelto
di seguire i seguenti step:

\begin{itemize}
  \item Analisi del \textbf{\textit{CV(Coefficente di Variazione)}} per
  eliminazione di parametri statisticamente insignificativi;
  \item \textbf{\textit{PCA(Principal Component Analysis)}} per la riduzione del
  numero di parametri e per l'eliminazione della correlazione tra essi;
  \item \textbf{\textit{Clustering}} per la riduzione del numero di esperimenti.
\end{itemize}


\subsection{Analisi del Coefficiente di Variazione}
In prima istanza è stata effettuata un'analisi sul coefficiente di variazione(COV)
il quale esprime quanta varianza contiene un parametro.\\
Quando il coefficiente di variazione è troppo piccolo il parametro corrispondente
non è statisticamente significativo, quindi in questa fase si eliminano i parametri
con COV nullo.\\
Nella figura si nota che non ci sono colonna con coefficiente di variazione nulla,
quindi tutti i parametri saranno utilizzati nelle successive analisi.
%inserire una figura

\subsection{PCA}
In questa fase è stata applicata la
\textbf{\textit{PCA(Principal Component Analysis)}}
la quale trasforma un workload con parametri correlati in uno contente parametri
non correlati.\\
L'utilizzo della PCA in questa è necessario anche per la fase successiva, in quanto
il clustering funziona meglio se i parametri non sono correlati.\\
Per effettuare la PCA si è fatto utilizzo del tool \textit{\textbf{JMP}} nella
figura è riportato l'output.\\
%figura
Facendo riferimento alla figura si può osservare che la prima componente del workload
conserva il x\% della varianza, quindi si è scelto di conservare l'x\% di varianza
saranno considerate le prima 6 componenti principali.\\
%figura
Nella figura sono evidenziati in rosso i parametri che hanno contribuito maggiormente,
in segno positivo o negativo, alla creazione delle componenti principali scelti.\\
In particolare:
\begin{itemize}
  \item \textbf{\textit{Principale 1:}}
  \item \textbf{\textit{Principale 2:}}
  \item \textbf{\textit{Principale 3:}}
  \item \textbf{\textit{Principale 4:}}
  \item \textbf{\textit{Principale 5:}}
  \item \textbf{\textit{Principale 6:}}
\end{itemize}

\subsection{Clustering}
In questa fase è stato effettuate un operazione di clustering, per poter ridurre
il numero di esperimenti, sul risultato ottenuto nello step precedente.\\
La tecnica di clasterizzazione scelta è di tipo gerarchico agglomerativo, in particolare
è stata utilizzata la metrica di work  per la creazione dei cluster.\\
In figura è riportato il dendogramma e la gerarchia di clusterizzazione.\\
%figura
Facendo riferimento alla figura precedente, si possono scegliere il numero di cluster,
posizionandosi nel ginocchio della curva rappresentante le distanze tra cluster.\\
In maniera analoga si può scegliere il numero di cluster utilizzando il criterio
clusterizzazione cubica riportato il figura.\\
%figura
Sulla base dei criteri sopra descritti si è scelto di considerare 6 cluster, per
ridurre il workload si è scelto di estrarre randomicamente un esperimento da ogni cluster.
\section{Conclusioni}
Dagli step descritti abbiamo ottenuto da un Workload reale uno sintetico,
ma non abbiamo preservato la varianza.\\
Per calcolare la varianza quanta varianza abbiamo conservato bisogna calcolare quanta
ne abbiamo conservato in ogni ogni step.\\
Per la PCA la varianza conservata è x\%, valore ottenuto da JMP, per il clustering
non si può fare un ragionamento basato sulla varianza ma bisogna utilizzare la devianza
poichè essa è indipendente dal grado di libertà dei cluster i quali hanno diverse
dimensioni.\\
Per il calcolo della percentuale di devianza conservata utilizzando il clustering
si è calcolata la devianza del workload sottoposto a PCA, in quanto il clustering è stato
effettuato successivamente, e poi è stata calcolata la devianza inter-cluster e intra-cluster,
poichè utilizzando il clustering si perde varianza inter-cluster e scegliendo un campione per
ogni cluster si perde varianza infra-cluster.
